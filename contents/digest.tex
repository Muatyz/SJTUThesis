% !TEX root = ../main.tex
%

\begin{digest}
    %颗粒介质是常见的强耗散非平衡态体系,具有如离散性、多体性、敏感性等丰富的物理特性。
    %由于颗粒固体的弹性模量相较于通常的结晶固体要小得多,因此也被 de Gennes 将其与玻璃、液晶等体系共同称为“软物质”,并评论称“Granular matter, in 1998, is at the level of solid-state physics in 1930.”
    %针对颗粒物理中丰富的物理性质,目前研究者们已开发出多角度的实验研究方法。
    %CT 成像、核磁共振等方式虽然已经成熟,但是不擅长于处理动态问题,而声学的高灵敏、非侵入特性能够弥补这一点。
    %从熟悉学习颗粒介质中的声学研究方法出发,本文研究了超声波在颗粒介质中的传播特性,还额外研究了颗粒介质的剪切响应性质。

    Granular media are common strongly dissipative non-equilibrium systems with rich physical properties, such as discrete, many-body, and sensitive to outside excitation.
    CT imaging, MRI, and other modalities, although mature in granular media research, are not so good at dealing with dynamic problems, which can be compensated by the highly sensitive, non-destructive nature of acoustic method.
    Starting from familiarizing with the acoustic research methods in granular media, this thesis paper studies the characteristics of ultrasonic waves propagation in granular media, and additionally studies the shear response properties.

    %超声波探测技术根据其探测思路可以大致分为两类:
    %1. 主动探测。超声波在颗粒介质中的传播是对其内部应力与接触结构的直接体现。超声波主动探测具有非侵入性、高灵敏度等优点;真实世界中的颗粒材料通常光学不透明或者剧烈光散射,使得超声波成为颗粒物质研究中独一无二的探针。
    %对经由颗粒介质传播后的声信号进行频谱、波形、声速、衰减、相似性参数等分析,有助于研究颗粒介质内部的结构/应力特征;
    %2. 被动探测。颗粒介质受剪切等外部激励时出现应力-应变曲线,这种宏观事件伴随着内部颗粒的重排、碰撞与摩擦等介观过程,这类机械过程会激发颗粒的机械振动,即发射出携带介观过程信息的声信号。这种振动事件被称为声发射(Acoustic Emission,AE)。
    %通过分析 AE 信号的频谱、能量、波形、振动模密度等特征可以对颗粒介质受应力时的蠕变(creeping)、滞滑(stick-slip)事件等特征过程进行研究。

    Ultrasonic detection techniques can be roughly categorized into two groups based on their detection ideas: 
    1. Active probing. Ultrasonic active probing has the advantages of being non-destructive and highly sensitive; granular materials in real world are often optically opaque or strongly light-scattering, which makes small amplitude ultrasonic waves a unique probe in the study of granular matter.
    The analysis of acoustic signals after propagation through granular media in terms of spectrum, waveform, speed of sound, attenuation, similarity parameters, etc., helps to study the configurational/stress characteristics inside the granular media;
    2. Passive probing. When applying external excitation like shearing or tapping to granular media, it will generate stress-strain curves, which are macroscopic events accompanied by mesoscopic processes such as internal particle rearrangement, collision, and friction. These mechanical processes will excite the vibration of the grains, that is, emit acoustic signals carrying mesoscopic process information. This vibration event is called acoustic emission (AE).
    By analyzing the spectrum, energy, waveform, vibration mode density and other characteristics of the AE signal, it is possible to study the characteristic processes such as creeping and stick-slip events when the granular medium is applied stress.
    
    %根据超声波的振幅和等效波长可以粗略划分出颗粒介质中的声学研究方式。
    %在等效波长远大于粒径时,此时关心的是在颗粒介质中的相干弹性波(coherent elastic waves)。
    %由于此时颗粒介质的力学响应可以类比于寻常固体,所以 EMT 采用仿射变换近似(affine approximation)、平均场近似等方法将其处理为等效的连续介质。此时关心的是通常是横纵波速 $V_{P}$、$V_{S}$ 与等效介质弹性模量 $K$、$G$ 之间存在的关系;
    %在等效波长与粒径数量级相近时,声波在颗粒介质中的传播表现出强烈的多重散射特征。颗粒接触点的耗散性使得声波在颗粒介质中的衰减吸收大大加强了,因此研究者提出了扩散行为近似、非线性波方程等模型来对颗粒介质进行研究。
    %在这些模型中,品质因子 $Q$、平均自由程 $l^{*}$ 等参数用于研究颗粒因接触点等因素产生的耗散性。在振幅较小时,超声波可作为颗粒介质内部应力结构的探针;振幅较大时,超声波则作为一种外部激励的能源。
    %典型的颗粒介质中的声信号可视为两部分:相干弹性波和多重散射尾波。相干波是由颗粒介质中复杂力链自平均形成的,因此对应力与接触构型不敏感,而散射波则是对接触构型非常敏感,这一点与相干波完全相反。

    Based on the amplitude and effective wavelength of the ultrasound waves, it is possible to roughly classify the way acoustic studies are performed in granular media.
    When the effective wavelength is much larger than the grain size, the concern at this point is coherent elastic waves in the granular medium.
    Since the mechanical response of the granular medium at this point is analogous to that of an ordinary solid, the effective medium theory uses affine approximation, mean field approximation, etc. to treat it as an effective continuous medium. The concern is usually the relationship between the transverse and longitudinal wave velocities $V_{P}$, $V_{S}$ and the elastic moduli $K$ and $G$ of the effective medium;
    The propagation of acoustic waves in granular media exhibits strong multiple scattering characteristics when the effective wavelength is of similar order of magnitude to the grain size. The dissipative nature of the grain contact points makes the attenuation absorption of acoustic waves in granular media greatly enhanced, so researchers have proposed models such as diffusive behavior approximation and nonlinear wave equations to study the granular media.
    In these models, physical quantities such as the quality factor $Q$ and the mean free path $l^{*}$ are used to study the dissipation of granular media due to factors such as contact points. At small amplitudes, ultrasound is used as a probe of the internal stress structure of the granular medium; at larger amplitudes(name finite-amplitude in acoutics), ultrasound waves are used as an externally excitation energy source. 
    The typical acoustic signal in a granular medium can be regarded as having two components: a coherent elastic wave part and a multiple scattering tail wave part. The coherent wave is formed by the self-averaging of complex force chains inside the granular medium and is therefore insensitive to the stress and contact configurations, whereas the scattered wave is very sensitive to the contact networks, which is the exact opposite of what the coherent wave behaves.
    
    %奈奎斯特-香农采样定理(Nyquist–Shannon sampling theorem)指出,对信号进行采样时,采样频率必须大于该信号最高频率的两倍,才能够重构出原始信号;如果要求不出现幅值失真,则采样频率还需要更高。
    %常用的声学线缆是阻抗 50 \unit{\ohm} 射频同轴线缆,传输带宽可达 \numrange{1}{20} \unit{\mega\hertz},确保了在对所关心频域内的声信号(<1 \unit{\mega\hertz})进行采样时不会失真。

    Nyquist-Shannon sampling theorem states that when sampling a signal, the sampling frequency must be greater than twice the highest frequency of the signal at least in order to reconstruct the original signal; if no amplitude distortion is required, the sampling frequency needs to be even higher. 
    Commonly used acoustic cables are 50 \unit{\ohm} impedance RF coaxial cables with a transmission bandwidth of up to \numrange{1}{20} \unit{\mega\hertz}, which ensures that acoustic signals (<1 \unit{\mega\hertz}) in the frequency range of interest are sampled without distortion.

    %为了研究颗粒介质的超声波传播特性,我们搭建了单轴应力容器声学系统,其由以下部件组成:
    %1. Tektronix Arbitrary Function Generator AFG31021。该仪器可以按照实验需求产生如控制频率的连续正弦波、设定循环数与触发时间间隔的正弦波列、单频脉冲等模拟电压信号,用于激励声学探头产生对应的机械振动;
    %2. Tektronix TBS2204B 示波器。该示波器通过 BNC 接口对可多达 $4$ 个独立模拟电压信号进行同步采集,可用于同时记录试探信号源以及在颗粒介质中传播后的响应信号。特别的是,示波器具有高精度采集和平均采集等多种采样模式,前者可用于记录瞬时事件信号,而后者则是通过 \numrange{16}{512} 次采集信号的线性平均极大地减少了热噪声的影响,从而得到相对平滑的信号曲线;
    %3. ATA-101B Power Amplifier。可以设置输入与输出的线缆阻抗(输入端可选取 50 \unit{\ohm}/1 \unit{\mega\ohm},输出端可选取 50 \unit{\ohm}/1.5 \unit{\ohm}。目前声学常用线缆都是 50 \unit{\ohm}),其主要功能是以最小步长 2 \unit{\decibel} 放大 AFG31021 产生的电压信号,从而满足使声学探头产生声压振幅的需求,通常适用于有限振幅波的非线性传播等实验中;
    %4. 清诚声发射 G150 与 W800 (成对)声学探头。声学探头在部分研究领域中又被称为换能器(Transducer),意指作为探头核心元件的压电陶瓷既能够将射频线缆传输的模拟电压信号通过电-压效应转换为机械振动的声信号,又能够感应机械振动的幅值强度、通过压-电效应将声信号转换为模拟电压信号,使其能被进一步展开信号处理与分析;
    % 5. 单轴应力圆柱容器。通过 SOLIDWORKS 软件设计,并采用厚度 \numlist{6;8;10} \unit{\milli\meter} 的亚克力板材料制作的容器(采用透明容器是为了兼容未来可能的 X 射线扫描建模需求)。容器内径为 D = 90 \unit{\milli\meter}, 最高支持厚度为 13 \unit{\centi\meter} 的颗粒介质进行实验。设置四柱单轴活塞以及用于放置金属块的加压台,从而对容器内的颗粒介质施加单轴应力。活塞与容器底部中心均留有 $\varphi$=19 \unit{\milli\meter} 的孔洞用于安装声学探头,具体交付图纸制作时考虑公差为 0.2 \unit{\milli\meter} 以便于灵活的更换。10 \unit{\milli\meter} 的亚克力板被弯曲为圆筒而制作为容器壁,因此可以安全承担实验需求范围内的应力。

    In order to study the ultrasonic propagation characteristics of granular media, we built a uniaxial stress vessel acoustic system, which consists of the following components: 1. Tektronix Arbitrary Function Generator AFG31021;2. Tektronix TBS2204B oscilloscope; 3. ATA-101B Power Amplifier; 4. Qing Cheng Acoustic Emission G150 and W800 acoustic probes(paired); 5. Single-axis stress cylindrical packagings. 

    %一般情况下作用于接收器表面的声压 $p(t)$ 都并不等于入射波的声压 $p_{i}(t)$,这是由探头硬质陶瓷圆面本身存在的声阻抗 $\widetilde{Z}_{S}(\omega)$ 所导致的。
    %在实验中,由于我们使用的是一对同工作频域的探头,面对面接触时可以简单视为声波经历了两次同样的声阻抗过程,且假定实验工作范围内使用的模拟电压信号幅值产生的压电效应满足线性。
    %为了测定探头的声阻抗,我们可以使用连续正弦电压信号对探头进行激励,并且将其理解为受迫阻尼振动过程,从而测定了声学探头的幅频特性曲线。
    %我们后续实验便可以借助该曲线来对实验中采集的响应信号进行修正。需要说明的是,声阻抗系数本身是复数,即探头本身对于信号会产生相位上的影响,但是如何准确测定这种相位差并且进行有效插值是一个较为困难的问题。我们在测定声阻抗系数的时候采用了稳态假设,而在实际实验的时候可能会由于弛豫时间的尺度较大而使得响应过程中伴随着过阻尼受迫振动的成分,因此在对实验所得的数据进行处理时,对于时间/相位相关的实验我们并不使用上述的修正方法,而是直接使用原始信号分析。

    In general the acoustic pressure $p(t)$ acting on the receiver surface is not equal to the acoustic pressure of the incident wave $p_{i}(t)$ due to the acoustic impedance $\widetilde{Z}_{S}(\omega)$ inherent in the hard ceramic circular surface of the probe. 
    In the experiment, since we use a pair of probes in the same operating frequency domain, face-to-face contact can simply be regarded as the acoustic wave undergoing the same acoustic impedance process twice, and it is assumed that the piezoelectric effect produced by the amplitude of the analog voltage signal used in the experimental operating range satisfies linearity. 
    In order to measure the acoustic impedance of the probe, we can determine the amplitude-frequency characteristic curve of the acoustic probe by exciting the probe using a continuous sinusoidal voltage signal at a same amplitude and interpreting it as a forced damped vibration process. 
    This curve can then be used in subsequent experiments to correct the response signal collected in the experiment. It should be noted that the acoustic impedance coefficients themselves are complex numbers, i.e., the probe itself has a phase effect on the response signal, but how to accurately determine this phase difference and interpolate it effectively is actually a difficult problem. We used the steady-state assumption in the determination of the acoustic impedance coefficients, while in the actual experiments the response may be accompanied by an overdamped forced vibration component due to the large scale of the relaxation time, so in the processing of the experimental data, the time/phase dependent experiments are not corrected as described above, and the raw signals are used directly for the analysis.

    %对于相干弹性波,本文讨论了如何使用飞行时间法和频散能量图法测定声速,并且将其和传统的时差法进行了比较。良好定义的波速应当满足在不同厚度下测得的结果相近,因此我们可以按照其确定在使用飞行时间法测定声速时的最佳信号参考点。识别信号得到各定义下的到达时间的算法思路是:
    %1. 信号平滑。由于示波器的采样频率通常很高,而且存在着一定的热噪声对信号进行干扰,因此我们借助 Matlab 的 movmean() 函数对信号进行线性权重的平滑处理;
    %2. 识别源信号(Src. = Source)的首峰时刻。由于系统硬件本身的问题仍然会导致源信号和响应信号存在一定量的偏置值,而这会对后续识别信号的波前位置造成影响;
    %3. 计算源信号的偏置值。取首峰幅值的 $1/80$ 作为其波前定义,通过 find() 函数找到其对应的时刻 $t_{O}$ 作为时间原点,即近似为源信号的发射时刻。取所记录的源信号中发射时刻前的信号均值作为源信号的偏置值,并将源信号减去该偏置值作为修正;
    %4. 识别响应信号(Res. = Response)的首峰时刻。从该步骤开始,需要通过控制寻峰算法中的 findpeaks() 函数中的峰最小半高宽(MinPeakWidth)和噪声阈值(MinPeakProminence),并且需要通过所识别的源信号波前时刻筛去响应信号中识别出的位置在其之前的干扰峰信号;
    %5. 识别响应信号的波前时刻。设定响应信号首峰幅值的 $1/10$ 作为其波前幅值,通过 find() 函数在源信号波前之后、响应信号首峰时刻之前的范围内寻找波前时刻。在这里使用的 $1/10$ 需要和响应信号的信噪比契合,如果信噪比过低就需要考虑调高这个比例,反之亦然;
    %6. 识别响应信号的首次过零时刻。在偏置修正完成后,寻找响应信号首峰时刻后的第一个过零点即可。波前与首次过零识别的准确性和响应信号的首峰识别准确性高度相关,如果首峰的信噪比足够高,便能确保以上步骤识别准确。
    
    For coherent elastic waves, this paper discusses how to determine the speed of sound using the time-of-flight and dispersive energy map methods and compares them with the conventional time-difference method. A well-defined wave velocity should satisfy the similarity of results measured at different thicknesses, so that we can follow it to determine the optimal signal reference point when using the time-of-flight method to determine the speed of sound. The idea of the algorithm to identify the signal to get the arrival time under each definition is:
    1. Signal smoothing. Since the sampling frequency of the oscilloscope is usually very high, and there is a certain amount of thermal noise that interferes with the signal, we use the linear weighted smoothing of the signal with the movmean() function provided by Matlab;
    2. Identify the arrival time of the first peak of the source signal. Due to the problems of the hardware itself, there will still be a certain amount of bias(or called "off-set" value) in the source and the response signal, which will affect the identification of the wavefront position of the signal;
    3. Calculate the bias value of the source signal. Take $1/80$ of the amplitude of the first peak as its wavefront definition, and find its corresponding moment $t_{O}$ as the absolute time origin through the find() function, which is approximated as the emission moment of the source signal. Take the average value of the signal before the emission moment of the recorded source signal as the bias value of the source signal, and subtract the bias value from the source signal as a correction; 
    4. Identify the arrival time of the first peak of the response signal. From this step on, it is necessary to control the minimum half-width of the peak ("MinPeakWidth") and the noise threshold ("MinPeakProminence") in the findpeaks() function of the peak-finding algorithm, and to filter out the interference peaks in the response signal that are identified before the wavefront time of the source signal;
    5. Identify the wavefront arrival time of the response signal. Set $1/10$ of the amplitude of the first peak of the response signal as its wavefront amplitude, and use the find() function to find the wavefront time in the range after the wavefront of the source signal and before the first peak time of the response signal. The $1/10$ used here needs to match the signal-to-noise ratio of the response signal. If the signal-to-noise ratio is too low, consider increasing this ratio, and vice versa;
    6. Identify the first zero-crossing time of the response signal. After the bias correction is completed, find the first zero-crossing point after the first peak time of the response signal. The accuracy of the wavefront and the first zero-crossing identification is highly correlated with the accuracy of the first peak identification of the response signal. If the signal-to-noise ratio of the first peak is high enough, the above steps can be ensured to be accurate.

    %本文探究了响应信号的波前到达时间、首峰到达时间和首次过零时间作为信号参考点时所测得飞行时间法定义的声速的不同,得出了以首峰到达时间为最佳参考信号点的结论,并且还总结了首峰随传播距离增大出现的展宽现象。
    %同时使用频散能量图法和时差法与飞行时间法测定的声速进行了比较。
    %各参考点的飞行时间法定义下的声速随着传播距离 $L$ 的增大向各自的时差法定义下的声速 $V_{L}$ 逼近,其中以波峰到达时间 $t_{1}$ 定义的声速在距离上的变化最为稳定。
    %因此在传播距离 $L$ 充分大时,我们可以使用飞行时间法替代时差法进行工程上的声速测量(因为时差法需要改变探头间的距离,而有的实验并不允许改变这一点,如与颗粒堆积结构本身相关的可逆性研究),而其中以波峰到达时间 $t_{1}$ 定义的又是最好的。
    
    This thesis paper explores the differences in the speed of sound defined by the time-of-flight method measured when the wavefront arrival time, the first peak arrival time, and the time of the first zero crossing of the response signal are used as the signal reference point, concludes that the first peak arrival time is the best reference signal point, and also summarizes the sound broadening phenomenon that occurs in the first peak as the propagation distance increases.
    Comparisons were also made between the speed of sound determined using the dispersive energy map method and the time-difference method with the time-of-flight one.
    The speed of sound defined by the time-of-flight method at each reference point approaches the speed of sound defined by the respective time-difference method $V_{L}$ as the propagation distance $L$ increases, with the speed of sound defined by the wave arrival time $t_{1}$ being the most stable over distance. 
    Thus for sufficiently large propagation distances $L$, we can use the time-of-flight method instead of the time-difference method for engineering sound velocity measurements (since the time-difference method requires changing the distance between the probes, and there are experiments that do not allow this to be changed, e.g., reversibility studies related to the granular packing structure itself), which again is best defined in terms of the wave arrival time $t_{1}$.

    %等效介质理论假定颗粒介质呈现几何学上统计性质的各向同性,各颗粒被均匀配位(接触点在球面上均匀分布),颗粒介质的体积足够大,且其中包含充分多的颗粒。我们检测了声速和介质所受单轴应力之间的指数关系为 1/8,并与等效介质理论的预测 1/6 进行对比和误差分析:
    %1. 真实的颗粒介质是异质性的。EMT 尝试通过增加颗粒系统的尺度与数量使得介质在统计上呈现各向同性,而在我们的实验中颗粒介质的约化厚度(Reduced Thickness)$L/d$ 并不高,约为 $7.33$;如果要在实验中做到这一点,那么作为探针的小振幅声波会因为颗粒介质因为尺度增加而增强的耗散性而剧烈衰减,使得响应信号的信噪比迅速降低。增加信号发生器的源振幅从而间接增加信噪比也是不可行的,因为在后文我们将讲述有关有限振幅波传播如何诱导颗粒介质内部接触与应力结构发生变化,即试探信号应当满足小振幅条件而不对颗粒介质造成泵浦效应;
    %2. 颗粒介质中颗粒的接触在其表面并不一定是均匀分布的;有实验和计算证明,对受循环剪切的颗粒介质进行 X 射线旋转扫描建模追踪,对于 $D=12\unit{\milli\meter}$ 的示踪颗粒,其上下平均接触数分别为 $8.41$、$6.56$。因此 EMT 论述过程中所需求的“接触点在单个颗粒表面呈现均匀分布”,在大多数情况下都是并不严格的假设;
    %3. 有效应力(effective pressure/stress)过低。借助标准大气压来定义有效应力,会发现在我们的实验中,颗粒介质的有效应力并不高(甚至有时是负数);而在成功重复出 EMT 预言的相关实验中,所施加应力往往都是 $\unit{\mega\pascal}$ 量级。既然声信号的传播可以粗糙理解为振动在颗粒介质内部力链上的传递,那么充分大的应力有助于颗粒介质的接触力更有效。

    Effective medium theory assumes that the granular medium is geometrically and statistically isotropic, that the particles are uniformly coordinated (contact points are uniformly distributed on the sphere), and that the granular medium is sufficiently large and contains a sufficient number of particles. We have examined the exponential relationship between the speed of sound and the uniaxial stress on the medium to be 1/8 and compared it with the prediction of 1/6 from the effective medium theory and analyzed the possible error sources:
    1. The real granular medium is heterogeneous; EMT tries to make the medium statistically isotropic by increasing the scale and grain number of granular systems, but the Reduced Thickness of the granular medium in our experiments is not very high at $L/d$, which is about $7.33$; if we were to do so in our experiments, the small-amplitude acoustic wave used as a probe would be drastically attenuated due to the enhanced dissipation of the granular medium due to the increase in scale, resulting in a rapid decrease in the signal-to-noise ratio of the response signal. If this were to be done experimentally, the small-amplitude acoustic waves used as probes would be violently attenuated by the enhanced dissipative properties of the granular medium due to the increased scale, causing the signal-to-noise ratio of the response signal to decrease rapidly.
    2. Contacts of particles in a granular medium are not necessarily uniformly distributed on the surface; it has been demonstrated experimentally and computationally that the average number of upper and lower contacts for a tracer particle with $D=12\unit{\milli\meter}$ is $8.41$ and $6.56$, respectively, for a tracer particle subjected to cyclic shear by rotational X-ray scanning modeling. 
    3. The effective stress is too low. Using standard atmospheric pressure to define effective stress, it is found that in our experiments, the effective stress of the granular medium is not high enough (sometimes even negative); and in successful experiments that have replicated the predictions of EMT, the applied stress is often on the order of $\unit{\mega\pascal}$. Since the propagation of sound signals can be roughly understood as the transfer of vibrations along force chains within the granular medium, a sufficiently large stress helps to make the contact forces of the granular medium more effective.

    %为了探讨展宽现象,使用归一化宽度 $W$ 对其进行描述,并且发现其与超声波在颗粒介质中的传播距离 $L$ 的指数关系,和一维随机层理论导出的结果符合较好。
    %可能存在的偏差来源有两种:
    %1. 波前到达时间的选取并不是绝对的。既然我们是通过 $S(t_{0}) = A_{1}\cdot k$($k\in(0,0.1]$)确定的 $t_{0}$,那么 $k$ 的具体数值会影响到 $t_{0}$ 的选取,从而影响到归一化宽度 $W$ 的测定;而在传播距离较大时,响应信号的信噪比会因为颗粒介质吸收、耗散引起的衰减而急剧降低,若要对经由任意传播距离的响应信号划定一条共通可行的 $k$ 实际上是并不容易的;
    %2. 理论推导中的 $n=2$ 是一维随机分层介质的情况。虽然我们已经将色散关系和衰减同时纳入考虑,并且实验模式中超声波的传播偏向于柱坐标中单个 $z$ 轴方向上,但是这一切仍不能改变“实际的颗粒介质仍然是三维体系”的事实,因此出现 $-0.448\sim-1/2$ 的偏差仍然是在理解范围内的。
    
    In order to explore the broadening phenomenon, the normalized width $W$ is used to describe it and it is found that its exponential relationship with the ultrasonic propagation distance $L$ in the granular medium is in good agreement with the results derived from the one-dimensional random layer theory. There are two possible sources of error:
    1. The selection of the wavefront arrival time is not absolute. Since we determine $t_{0}$ through $S(t_{0}) = A_{1}\cdot k$ ($k\in(0,0.1]$), the specific value of $k$ will affect the selection of $t_{0}$, and thus affect the determination of the normalized width $W$; and as the propagation distance increases, the signal-to-noise ratio of the response signal will decrease sharply due to the attenuation caused by the absorption and dissipation of the granular medium, and it is not easy to define a common feasible $k$ for the response signal propagated through any distance;
    2. The $n=2$ in the theoretical derivation is for the case of one-dimensional random layered media. Although we have taken into account both the dispersion relationship and the attenuation, and the experimental mode of ultrasonic wave propagation tends to be in the single $z$-axis direction in cylindrical coordinates, all this does not change the fact that "the actual granular medium is still a three-dimensional system", so the deviation of $-0.448\sim-1/2$ is still within the range of understanding.

    %对于散射尾波,本文探讨了由辐射传递方程导出的扩散行为近似和尝试使用非线性波方程来对颗粒介质中的超声尾波进行描述。并且我们发现其与测定的强度曲线符合较好,通过拟合计算得到了平均自由程 $l^{*}$ 和非弹性吸收时间 $\tau_{\alpha}$。
    %我们的一般处理思路是:
    %1. 按照前文中所陈述的方法测定飞行时间法定义下的声速作为能量传输速度的近似值,同时计算信号的偏置值并且对其进行修正;
    %2. 计算声源能量。对于方波信号,其能量即为其幅值平方在时域上的积分。需要注意的是探头本身由于频响特性曲线会对该能量进行耗散,因此后续在拟合的过程中还需要额外引入系数进行修正;
    %3. 计算非弹性吸收时间 $\tau_{\alpha}$。首先在时域上截取散射尾波(即需要去除频率成分相对较低的相干首波,避免其幅值影响后续的信号处理);然后使用 movemean() 函数对响应信号尾波进行平滑处理,然后对其求平方并除以声阻抗 50\unit{\ohm} 得到时域上的功率波形。通过 envelope() 函数对功率波形求包络线,即得到了近似的 $J(t)$ 函数图像。在时间足够长的情况下,即有 $J(t)\propto {\ee}^{-t/\tau_{\alpha}}$,因此对其进行对数处理并求出斜率,即可得到非弹性吸收时间;
    %4. 拟合计算平均自由程 $l^{*}$。由于解本身是无穷级数,在实际应用时我们需要对其进行截断。在尝试拟合过程中,我们发现在 $n>10$ 时就已经可以得到较好的曲线,在实际计算中我们取级数 N=200。由于我们已经通过其它方式得到了方程中的各参数,因此实际拟合时的自由参数只有 $l^{*}$,从而极大地减少了拟合复杂度(如果不预先计算出参量,拟合探索过程所需要的循环数将轻易超过 Matlab 允许的循环上限)。
    
    For the scattering tail, this paper explores the diffusive behavior approximation derived from the radiative transfer equation and attempts to use the nonlinear wave equation to characterize the ultrasonic tail in granular media. And we find that it is in good agreement with the measured intensity profile, and the mean free range $l^{*}$ and inelastic absorption time $\tau_{\alpha}$ are calculated by fitting.
    Our general approach to the idea is:
    1. Determine the speed of sound defined by the time-of-flight method as an approximate value of the energy transfer speed as stated in the previous paragragh, and calculate the bias value of the signal and correct it;
    2. Calculate the source energy. For a square wave signal, its energy is the square of its amplitude integrated in the time domain. It should be noted that the probe itself dissipates the energy due to the frequency response curve, so an additional coefficient is needed for correction in the subsequent fitting process;
    3. Calculate the inelastic absorption time $\tau_{\alpha}$. First, the scattering tail wave is truncated in the time domain (i.e., the coherent first wave with relatively low frequency components is removed to avoid its amplitude affecting the subsequent signal processing); then the response tail wave is smoothed using the movemean() function, and then squared and divided by the acoustic impedance 50\unit{\ohm} to obtain the power waveform in the time domain. The envelope() function is used to obtain the approximate $J(t)$ function image by taking the envelope of the power waveform. In the case of a sufficiently long time, $J(t)\propto {\ee}^{-t/\tau_{\alpha}}$, so by taking the logarithm and calculating the slope, the inelastic absorption time can be obtained;
    4. Fit and calculate the mean free range $l^{*}$. Since the solution itself is an infinite series, it needs to be truncated in practical applications. In the fitting process, we find that a good curve can be obtained when $n>10$, and in actual calculations we take N=200. Since we have already obtained the various parameters in the equation by other means, the actual free parameter in the fitting is only $l^{*}$, which greatly reduces the complexity of the fitting (if the parameters are not calculated in advance, the number of loops required for the fitting exploration process will easily exceed the loop limit allowed in Matlab).

    %研究了振幅和颗粒介质非线性的关系,引入两信号的归一化交叉关联函数作为相似性参数来比较颗粒介质的内部应力结构是否发生变化,并且使用 Parseval 定理论证了其在频域上的推广与时域下的定义等价。
    %使用连续激励协议对颗粒介质发射声信号,发现振幅超过一定值后使得其结构因泵浦而变化,并引入相似性参数对其进行描述;通过带通滤波器来对响应信号各谐波分量进行提取,并且观察其与激励源振幅的对应幂律是否继续成立。
    %观察到了相似性参数在振幅约为 9.14 \unit{\volt} 和在泵浦结束时的首个试探信号都出现了骤降,随后相似性参数均趋近于 1,说明颗粒介质内部结构因为外部声学激励而发生了变化;声速的相对变化计算来源于小振幅时所测声速的平均值,而在逐渐激励的过程中有逐渐降低的趋势,这说明了颗粒介质的模量因为连续激励而有所降低。
    %在小于约 3.95 \unit{\volt} 时,基波和一次谐波都分别与激励振幅的一次和二次呈现良好的线性符合,而在大于该阈值后出现了偏离,这也能说明声学激励对于颗粒介质的衰减系数造成了影响。

    The relationship between the amplitude and the nonlinearity of the granular medium is investigated, the normalized cross-correlation function of the two signals is introduced as a similarity parameter to compare whether the internal stress structure of the granular medium has changed or not, and its generalization in the frequency domain is demonstrated to be equivalent to the definition in the time domain using Parseval's theorem.
    The granular medium is excited by a continuous excitation protocol, and it is found that the structure changes due to pumping when the amplitude exceeds a certain value, and a similarity parameter is introduced to describe it; the response signal is extracted by a bandpass filter to observe whether the corresponding power law of the response signal harmonics continues to hold with the source amplitude.
    It is observed that the similarity parameter drops sharply at an amplitude of about 9.14 \unit{\volt} and at the first probing signal at the end of the pumping, and then the similarity parameter tends to 1, indicating that the internal structure of the granular medium has changed due to external acoustic excitation; the relative change in sound speed is calculated from the average value of the sound speed measured at small amplitudes, and there is a gradual decrease in the process of gradual excitation, indicating that the modulus of the granular medium has decreased due to continuous excitation.
    When the amplitude is less than about 3.95 \unit{\volt}, the fundamental and first harmonic are both in good linear agreement with the first and second powers of the excitation amplitude, respectively, and after the threshold is exceeded, deviations occur, which also indicates that the acoustic excitation has an effect on the attenuation coefficient of the granular medium.


    %最后对颗粒介质的剪切响应进行了研究,发现了无法向应力的随机稀疏堆积和施加法向应力的随机密集堆积应力-应变曲线的不同。我们可以按照应力降(Shear Stress Drop,SSD)的大小来对滞滑事件进行粗分类。类比于环形剪切相关实验中的方法,我们按照应力降的幅值将滞滑事件分为三类:微滞滑(micro-,$\text{SSD}<0.2$ \unit{\newton})、小滞滑(minor-,0.2 \unit{\newton}$<\text{SSD}<0.4$ \unit{\newton})和主滞滑(major-,在地震学也被称作 failure,$\text{SSD}>0.4$ \unit{\newton})。
    %我们统计了微滞滑事件的数量以及其在最近主滞滑事件前的时间间隔,从而得到了通过微滞滑事件计数来预测可能的滞滑失效事件的事件。我们对微滞滑的发生数量进行了对数处理后绘制了图像。其解读方式是,在发生了一次主滞滑事件后,对微滞滑进行计数,若已经发生了 $N$ 次,则下一次主滞滑可能发生的时间间隔为 $T(log{(N)})$ \unit{s}。在原点附近出现了拐点,推测是因为力学传感器本身会出现一定的分辨率程度的波动($\sim$ 0.1 \unit{\newton}),使得微滞滑的计数偏多。在未来引入多通道采集卡,从而通过声学手段辅助分析剪切过程中的滞滑事件,将有助于得到更理想的图像结果。
    
    Finally, the shear response of the granular medium is investigated and the difference between the stress-strain curves of random loose packing without normal stress and random close packing with applied normal stress is found. We can coarsely categorize the hysteresis events according to the magnitude of the Shear Stress Drop (SSD). Analogous to the methods used in the ring shear correlation experiments, we classify the stalling events into three categories according to the magnitude of the stress drop: micro-slip ($\text{SSD}<0.2$ \unit{\newton}), minor-slip (0.2 \unit{\newton}$<\text{SSD}<0.4$ \unit{ \newton}), major slip (also known as failure in seismology, $\text{SSD}>0.4$ \unit{\newton}).
    We counted the number of microslip events and their time intervals before the most recent main slip event, thus obtaining events that predict possible slip failure events by counting microslip events. We plotted the images after logarithmic processing of the number of micro-stall slip occurrences. This is interpreted by counting the micro-lag slips after a primary lag slip event has occurred, and if it has occurred $N$ times, the time interval at which the next primary lag slip may occur is $T(log{(N)})$ \unit{s}. The inflection point near the origin is presumed to be due to the fact that the mechanics sensor itself fluctuates with a certain degree of resolution ($\sim$ 0.1 \unit{\newton}), which skews the counts of the micro-stall slips. In the future, the introduction of a multi-channel acquisition card, thus assisting in the analysis of the hysteresis events during shear by acoustic means, will help to obtain more satisfactory image results.
    
    %这些技术积累将为后续使用声学研究颗粒介质中的剪切带、监测受剪切过程的声发射事件以及将声学方法耦合至 CT 成像研究等提供基础。
    The accumulation of these techniques will provide the basis for the subsequent use of acoustics to study shear zones in granular media, to monitor acoustic emission events subject to shear processes, and to couple acoustic methods to CT imaging studies.

\end{digest}
