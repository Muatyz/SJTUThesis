% !TeX root = ../main.tex

\chapter{模板说明书}

\section{脚注}

上海交通大学 \footnote{上海交通大学}

\section{字体}



{\songti 上海交通大学}

{\heiti 上海交通大学}

{\ifcsname fangsong\endcsname\fangsong\else[无 \cs{fangsong} 字体。]\fi 上海交通大学}

{\ifcsname kaishu\endcsname\kaishu\else[无 \cs{kaishu} 字体。]\fi 上海交通大学}

\section{数学}

\subsection{数字和单位}

宏包 \pkg{siunitx} 提供了更好的数字和单位支持:
\begin{itemize}
  \item \num{12345.67890}
  \item \complexnum{1+-2i}
  \item \num{.3e45}
  \item \numproduct{1.654 x 2.34 x 3.430}
  \item \unit{kg.m.s^{-1}}
  \item \unit{\micro\meter} $\unit{\micro\meter}$
  \item \unit{\ohm} $\unit{\ohm}$
  \item \numlist{10;20}
  \item \numlist{10;20;30}
  \item \qtylist{0.13;0.67;0.80}{\milli\metre}
  \item \numrange{10}{20}
  \item \qtyrange{10}{20}{\degreeCelsius}
\end{itemize}

\subsection{数学符号和公式}

按照国标 GB/T 3102.11—1993《物理科学和技术中使用的数学符号》,
微分符号 $\dd$ 应使用直立体。除此之外,数学常数也应使用直立体:
\begin{itemize}
  \item 微分符号 $\dd$:\cs{dd}
  \item 圆周率 $\uppi$:\cs{uppi}
  \item 自然对数的底 $\ee$:\cs{ee}
  \item 虚数单位 $\ii$, $\jj$:\cs{ii} \cs{jj}
\end{itemize}

公式应另起一行居中排版。公式后应注明编号,按章顺序编排,编号右端对齐。
\begin{equation}
  \ee^{\ii\uppi} + 1 = 0,
\end{equation}
\begin{equation}
  \frac{\dd^2 u}{\dd t^2} = \int f(x) \dd x.
\end{equation}

公式末尾是需要添加标点符号的,至于用逗号还是句号,取决于公式下面一句是接着公式说的,还是另起一句。
\begin{equation}
  \frac{2h}{\pi}\int_{0}^{\infty}\frac{\sin\left( \omega\delta \right)}{\omega}
  \cos\left( \omega x \right) \dd\omega = 
  \begin{cases}
    h, & \left| x \right| < \delta, \\
    \frac{h}{2}, & x = \pm \delta, \\
    0, & \left| x \right| > \delta.
  \end{cases}
\end{equation}
公式较长时最好在等号“$=$”处转行。
\begin{align}
    & I (X_3; X_4) - I (X_3; X_4 \mid X_1) - I (X_3; X_4 \mid X_2) \nonumber \\
  = & [I (X_3; X_4) - I (X_3; X_4 \mid X_1)] - I (X_3; X_4 \mid \tilde{X}_2) \\
  = & I (X_1; X_3; X_4) - I (X_3; X_4 \mid \tilde{X}_2).
\end{align}

如果在等号处转行难以实现,也可在 $+$、$-$、$\times$、$\div$ 运算符号处转行,转行
时运算符号仅书写于转行式前,不重复书写。
\begin{multline}
  \frac{1}{2} \Delta (f_{ij} f^{ij}) =
    2 \left(\sum_{i<j} \chi_{ij}(\sigma_{i} - \sigma_{j})^{2}
    + f^{ij} \nabla_{j} \nabla_{i} (\Delta f) \right. \\
  \left. + \nabla_{k} f_{ij} \nabla^{k} f^{ij} +
    f^{ij} f^{k} \left[2\nabla_{i}R_{jk}
    - \nabla_{k} R_{ij} \right] \vphantom{\sum_{i<j}} \right).
\end{multline}

\subsection{定理环境}

示例文件中使用 \pkg{ntheorem} 宏包配置了定理、引理和证明等环境。用户也可以使用
\pkg{amsthm} 宏包。

这里举一个“定理”和“证明”的例子。
\begin{theorem}[留数定理]
\label{thm:res}
  假设 $U$ 是复平面上的一个单连通开子集,$a_1, \ldots, a_n$ 是复平面上有限个点,
  $f$ 是定义在 $U \backslash \{a_1, \ldots, a_n\}$ 上的全纯函数,如果 $\gamma$
  是一条把 $a_1, \ldots, a_n$ 包围起来的可求长曲线,但不经过任何一个 $a_k$,并且
  其起点与终点重合,那么:

  \begin{equation}
    \label{eq:res}
    \oint\limits_\gamma f(z)\, \dd z = 2\uppi \ii \sum_{k=1}^n \operatorname{I}(\gamma, a_k) \operatorname{Res}(f, a_k).
  \end{equation}

  如果 $\gamma$ 是若尔当曲线,那么 $\operatorname{I}(\gamma, a_k) = 1$,因此:

  \begin{equation}
    \label{eq:resthm}
    \oint\limits_\gamma f(z)\, \dd z = 2\uppi \ii \sum_{k=1}^n \operatorname{Res}(f, a_k).
  \end{equation}

  在这里,$\operatorname{Res}(f, a_k)$ 表示 $f$ 在点 $a_k$ 的留数,
  $\operatorname{I}(\gamma, a_k)$ 表示 $\gamma$ 关于点 $a_k$ 的卷绕数。卷绕数是
  一个整数,它描述了曲线 $\gamma$ 绕过点 $a_k$ 的次数。如果 $\gamma$ 依逆时针方
  向绕着 $a_k$ 移动,卷绕数就是一个正数,如果 $\gamma$ 根本不绕过 $a_k$,卷绕数
  就是零。

  定理~\ref{thm:res} 的证明。

  \begin{proof}
    首先,由……

    其次,……

    所以……
  \end{proof}
\end{theorem}

\section{引用文献的标注}

按照教务处的要求,参考文献外观应符合国标 GB/T 7714 的要求。模版使用 \BibLaTeX{}
配合 \pkg{biblatex-gb7714-2015} 样式包%
\footnote{\url{https://www.ctan.org/pkg/biblatex-gb7714-2015}}%
控制参考文献的输出样式,后端采用 \pkg{biber} 管理文献。

请注意 \pkg{biblatex-gb7714-2015} 宏包 2016 年 9 月才加入 CTAN,如果你使用的
\TeX{} 系统版本较旧,可能没有包含 \pkg{biblatex-gb7714-2015} 宏包,需要手动安装。
\BibLaTeX{} 与 \pkg{biblatex-gb7714-2015} 目前在活跃地更新,为避免一些兼容性问
题,推荐使用较新的版本。

正文中引用参考文献时,使用 \verb|\cite{key1,key2,key3...}| 可以产生“上标引用的参
考文献”,如 \cite{Yu2001,Cheng1999,LSC1957}。使用
\verb|\parencite{key1,key2,key3...}| 则可以产生水平引用的参考文献,例如
\parencite{Li1999,Jiang1989,Hopkinson1999}。请看下面的例子,将会穿插使用水平的和
上标的参考文献:普通图书\parencite{Yu2001,Jiang1998},论文集、会议录
\cite{CSTAM1990},科技报告\parencite{WHO1970},学位论文\cite{Zhang1998},专利文
献\parencite{Jiang1989,HBLZ2001},专著中析出的文献\cite{Cheng1999,GBT2659},期刊
中析出的文献\parencite{Li1999,Li2000},报纸中析出的文献\cite{Ding2000}, 电子文献
\parencite{Jiang1999,Christine1998,Xiao2001}。

可以使用 \verb|\nocite{key1,key2,key3...}| 将参考文献条目加入到文献表中但不在正
文中引用。使用 \verb|\nocite{*}| 可以将参考文献数据库中的所有条目加入到文献表
中。
\nocite{Yang1999,Schinstock2000,Wen1990,GBT16159}

\section{插图}

插图功能是利用 \TeX{} 的特定编译程序提供的机制实现的,不同的编译程序支持不同的图
形方式。有的同学可能听说“\LaTeX{} 只支持 EPS”,事实上这种说法是不准确的。\XeTeX{}
可以很方便地插入 EPS、PDF、PNG、JPEG 格式的图片。

一般图形都是处在浮动环境中。之所以称为浮动是指最终排版效果图形的位置不一定与源文
件中的位置对应,这也是刚使用 \LaTeX{} 同学可能遇到的问题。如果要强制固定浮动图形
的位置,请使用 \pkg{float} 宏包,它提供了 \texttt{[H]} 参数。

\subsection{单个图形}

图要有图题,研究生图题采用中英文对照,并置于图的编号之后,图的编号和图题应置于图
下方的居中位置。引用图应在图题右上角标出文献来源。文中必须有关于本插图的提示,如
“见图~\ref{fig:energy-distrib}”、“如图~\ref{fig:energy-distrib} 所示”等。该页空
白不够排写该图整体时,则可将其后文字部分提前排写,将图移到次页。

\begin{figure}[!htp]
  \centering
  \begin{tikzpicture}
    \begin{axis}[
      width=12cm,
      height=9cm,
      xmin=0, xmax=7,
      xlabel={$r$ (\unit{\milli\metre})},
      ymin=-1000, ymax=11000,
      ylabel={Energy (\unit[per-mode=symbol]{\watt\per\cubic\metre})},
      scaled ticks=false,
      tick label style={
        /pgf/number format/1000 sep=,
        font={\zihao{-5}},
      },
      minor tick num=1,
      tick pos=left,
      tick align=outside,
      tick style={thin,black},
    ]
      \addplot [only marks,mark=square*] 
        table [x={radial}, y={energy}, col sep=comma] 
        {./assets/energy-distrib.csv};
      \node at (2,6000) 
        {$q_{v}=\dfrac{\sigma\omega^{2}|\mathbf{A}|^{2}}{2}$};
    \end{axis}
  \end{tikzpicture}
  \bicaption{内热源沿径向的分布}{Energy distribution along radial}
  \label{fig:energy-distrib}
\end{figure}

\subsection{多个图形}

简单插入多个图形的例子如图~\ref{fig:SRR} 所示。这两个水平并列放置的子图共用一个
图形计数器,没有各自的子图题。

\begin{figure}[!htp]
  \centering
  \includegraphics[height=2cm]{sjtu-vi-badge-blue.pdf}
  \hspace{1cm}
  \includegraphics[height=2cm]{sjtu-vi-badge-blue.pdf}
  \bicaption{中文题图}{English caption}
  \label{fig:SRR}
\end{figure}

如果多个图形相互独立,并不共用一个图形计数器,那么用 \texttt{minipage} 或者
\texttt{parbox} 就可以,如图~\ref{fig:parallel1} 与图~\ref{fig:parallel2}。

\begin{figure}[!htp]
  \centering
  \begin{minipage}{0.48\textwidth}
    \centering
    \includegraphics[height=1.5cm]{sjtu-vi-name-blue.pdf}
    \caption{并排第一个图}
    \label{fig:parallel1}
  \end{minipage}\hfill
  \begin{minipage}{0.48\textwidth}
    \centering
    \includegraphics[height=1.5cm]{sjtu-vi-name-blue.pdf}
    \caption{并排第二个图}
    \label{fig:parallel2}
  \end{minipage}
\end{figure}

如果要为共用一个计数器的多个子图添加子图题,建议使用较新的 \pkg{subcaption}宏
包,不建议使用 \pkg{subfigure} 或 \pkg{subfig} 等宏包。

推荐使用 \pkg{subcaption} 宏包的 \cs{subcaptionbox} 并排子图,子图题置于子图之
下,子图号用 a)、b) 等表示。也可以使用 \pkg{subcaption} 宏包的 \cs{subcaption}
(放在 minipage中,用法同 \cs{caption})。

\pkg{subcaption} 宏包也提供了 \pkg{subfigure} 和 \pkg{subtable} 环境,如
图~\ref{fig:subfigure}。

\begin{figure}[!htp]
  \centering
  \begin{subfigure}{0.3\textwidth}
    \centering
    \includegraphics[height=2cm]{sjtu-vi-badge-blue.pdf}
    \caption{校徽}
  \end{subfigure}
  \hspace{1cm}
  \begin{subfigure}{0.4\textwidth}
    \centering
    \includegraphics[height=1.5cm]{sjtu-vi-name-blue.pdf}
    \caption{校名。注意这个图略矮些,subfigure 中同一行的子图在顶端对齐。}
  \end{subfigure}
  \caption{包含子图题的范例(使用 subfigure)}
  \label{fig:subfigure}
\end{figure}

搭配 \pkg{bicaption} 宏包时,可以启用 \cs{subcaptionbox} 和 \cs{subcaption} 的双
语变种 \cs{bisubcaptionbox} 和 \cs{bisubcaption},如图~\ref{fig:bisubcaptionbox}
所示。

\begin{figure}[!hbtp]
  \centering
  \bisubcaptionbox{$R_3 = 1.5\text{mm}$ 时轴承的压力分布云图}%
                  {Pressure contour of bearing when $R_3 = 1.5\text{mm}$}%
                  [6.4cm]{\includegraphics[height=3cm]{example-image-a.pdf}}
  \hspace{1cm}
  \bisubcaptionbox{$R_3 = 2.5\text{mm}$ 时轴承的压力分布云图}%
                  {Pressure contour of bearing when $R_3 = 2.5\text{mm}$}%
                  [6.4cm]{\includegraphics[height=3cm]{example-image-b.pdf}}
  \bicaption{包含子图题的范例(使用 subcaptionbox)}
            {Example with subcaptionbox}
  \label{fig:bisubcaptionbox}
\end{figure}


\section{表格}

\subsection{基本表格}

编排表格应简单明了,表达一致,明晰易懂,表文呼应、内容一致。表题置于表上,研究生
学位论文可以用中、英文两种文字居中排写,中文在上,也可以只用中文。

表格的编排建议采用国际通行的三线表\footnote{三线表,以其形式简洁、功能分明、阅读
方便而在科技论文中被推荐使用。三线表通常只有 3 条线,即顶线、底线和栏目线,没有
竖线。}。三线表可以使用 \pkg{booktabs} 提供的 \cs{toprule}、\cs{midrule} 和
\cs{bottomrule}。它们与 \pkg{longtable} 能很好的配合使用。

\begin{table}[!hpt]
  \caption[一个颇为标准的三线表]{一个颇为标准的三线表\footnotemark}
  \label{tab:firstone}
  \centering
  \begin{tabular}{@{}llr@{}} \toprule
    \multicolumn{2}{c}{Item} \\ \cmidrule(r){1-2}
    Animal & Description & Price (\$)\\ \midrule
    Gnat  & per gram  & 13.65 \\
          & each      & 0.01 \\
    Gnu   & stuffed   & 92.50 \\
    Emu   & stuffed   & 33.33 \\
    Armadillo & frozen & 8.99 \\ \bottomrule
  \end{tabular}
\end{table}
\footnotetext{这个例子来自
  \href{https://mirrors.sjtug.sjtu.edu.cn/ctan/macros/latex/contrib/booktabs/booktabs.pdf}%
  {《Publication quality tables in LaTeX》}(\pkg{booktabs} 宏包的文档)。这也是
  一个在表格中使用脚注的例子,请留意与 \pkg{threeparttable} 实现的效果有何不
  同。}

\subsection{复杂表格}

我们经常会在表格下方标注数据来源,或者对表格里面的条目进行解释。可以用
\pkg{threeparttable} 实现带有脚注的表格,如表~\ref{tab:footnote}。

\begin{table}[!htpb]
  \bicaption{一个带有脚注的表格的例子}{A Table with footnotes}
  \label{tab:footnote}
  \centering
  \begin{threeparttable}[b]
     \begin{tabular}{ccd{4}cccc}
      \toprule
      \multirow{2}*{total} & \multicolumn{2}{c}{20\tnote{a}} & \multicolumn{2}{c}{40} & \multicolumn{2}{c}{60} \\
      \cmidrule(lr){2-3}\cmidrule(lr){4-5}\cmidrule(lr){6-7}
      & www & \multicolumn{1}{c}{k} & www & k & www & k \\ % 使用说明符 d 的列会自动进入数学模式,使用 \multicolumn 对文字表头做特殊处理
      \midrule
      & $\underset{(2.12)}{4.22}$ & 120.0140\tnote{b} & 333.15 & 0.0411 & 444.99 & 0.1387 \\
      & 168.6123 & 10.86 & 255.37 & 0.0353 & 376.14 & 0.1058 \\
      & 6.761    & 0.007 & 235.37 & 0.0267 & 348.66 & 0.1010 \\
      \bottomrule
    \end{tabular}
    \begin{tablenotes}
    \item [a] the first note.
    \item [b] the second note.
    \end{tablenotes}
  \end{threeparttable}
\end{table}

如某个表需要转页接排,可以用 \pkg{longtable} 实现。接排时表题省略,表头应重复书
写,并在右上方写“续表 xx”,如表~\ref{tab:performance}。

\begin{ThreePartTable}
  \begin{TableNotes}
    \item[a] 一个脚注
    \item[b] 另一个脚注
  \end{TableNotes}
  \begin{longtable}[c]{c*{6}{r}}
    \bicaption{实验数据}{Experimental data}
    \label{tab:performance} \\
    \toprule
    测试程序 & \multicolumn{1}{c}{正常运行} & \multicolumn{1}{c}{同步}
      & \multicolumn{1}{c}{检查点} & \multicolumn{1}{c}{卷回恢复}
      & \multicolumn{1}{c}{进程迁移} & \multicolumn{1}{c}{检查点} \\
    & \multicolumn{1}{c}{时间 (s)} & \multicolumn{1}{c}{时间 (s)}
      & \multicolumn{1}{c}{时间 (s)} & \multicolumn{1}{c}{时间 (s)}
      & \multicolumn{1}{c}{时间 (s)} &  文件(KB)\\
    \midrule
    \endfirsthead
    \multicolumn{7}{l}{\textbf{续表~\thetable}} \\
    % 英语论文:\multicolumn{7}{r}{\textbf{Table~\thetable~(continued)}} \\
    \toprule
    测试程序 & \multicolumn{1}{c}{正常运行} & \multicolumn{1}{c}{同步}
      & \multicolumn{1}{c}{检查点} & \multicolumn{1}{c}{卷回恢复}
      & \multicolumn{1}{c}{进程迁移} & \multicolumn{1}{c}{检查点} \\
    & \multicolumn{1}{c}{时间 (s)} & \multicolumn{1}{c}{时间 (s)}
      & \multicolumn{1}{c}{时间 (s)} & \multicolumn{1}{c}{时间 (s)}
      & \multicolumn{1}{c}{时间 (s)}&  文件(KB)\\
    \midrule
    \endhead
    \hline
    \multicolumn{7}{r}{续下页}
    \endfoot
    \insertTableNotes
    \endlastfoot
    CG.A.2 & 23.05 & 0.002 & 0.116 & 0.035 & 0.589 & 32491 \\
    CG.A.4 & 15.06 & 0.003 & 0.067 & 0.021 & 0.351 & 18211 \\
    CG.A.8 & 13.38 & 0.004 & 0.072 & 0.023 & 0.210 & 9890 \\
    CG.B.2 & 867.45 & 0.002 & 0.864 & 0.232 & 3.256 & 228562 \\
    CG.B.4 & 501.61 & 0.003 & 0.438 & 0.136 & 2.075 & 123862 \\
    CG.B.8 & 384.65 & 0.004 & 0.457 & 0.108 & 1.235 & 63777 \\
    MG.A.2 & 112.27 & 0.002 & 0.846 & 0.237 & 3.930 & 236473 \\
    MG.A.4 & 59.84 & 0.003 & 0.442 & 0.128 & 2.070 & 123875 \\
    MG.A.8 & 31.38 & 0.003 & 0.476 & 0.114 & 1.041 & 60627 \\
    MG.B.2 & 526.28 & 0.002 & 0.821 & 0.238 & 4.176 & 236635 \\
    MG.B.4 & 280.11 & 0.003 & 0.432 & 0.130 & 1.706 & 123793 \\
    MG.B.8 & 148.29 & 0.003 & 0.442 & 0.116 & 0.893 & 60600 \\
    LU.A.2 & 2116.54 & 0.002 & 0.110 & 0.030 & 0.532 & 28754 \\
    LU.A.4 & 1102.50 & 0.002 & 0.069 & 0.017 & 0.255 & 14915 \\
    LU.A.8 & 574.47 & 0.003 & 0.067 & 0.016 & 0.192 & 8655 \\
    LU.B.2 & 9712.87 & 0.002 & 0.357 & 0.104 & 1.734 & 101975 \\
    LU.B.4 & 4757.80 & 0.003 & 0.190 & 0.056 & 0.808 & 53522 \\
    LU.B.8 & 2444.05 & 0.004 & 0.222 & 0.057 & 0.548 & 30134 \\
    EP.A.2 & 123.81 & 0.002 & 0.010 & 0.003 & 0.074 & 1834 \\
    EP.A.4 & 61.92 & 0.003 & 0.011 & 0.004 & 0.073 & 1743 \\
    EP.A.8 & 31.06 & 0.004 & 0.017 & 0.005 & 0.073 & 1661 \\
    EP.B.2 & 495.49 & 0.001 & 0.009 & 0.003 & 0.196 & 2011 \\
    EP.B.4 & 247.69 & 0.002 & 0.012 & 0.004 & 0.122 & 1663 \\
    EP.B.8 & 126.74 & 0.003 & 0.017 & 0.005 & 0.083 & 1656 \\
    SP.A.2 & 123.81 & 0.002 & 0.010 & 0.003 & 0.074 & 1854 \\
    SP.A.4 & 51.92 & 0.003 & 0.011 & 0.004 & 0.073 & 1543 \\
    SP.A.8 & 31.06 & 0.004 & 0.017 & 0.005 & 0.073 & 1671 \\
    SP.B.2 & 495.49 & 0.001 & 0.009 & 0.003 & 0.196 & 2411 \\
    SP.B.4 \tnote{a} & 247.69 & 0.002 & 0.014 & 0.006 & 0.152 & 2653 \\
    SP.B.8 \tnote{b} & 126.74 & 0.003 & 0.017 & 0.005 & 0.082 & 1755 \\
    \bottomrule
  \end{longtable}
\end{ThreePartTable}

\section{算法环境}

算法环境可以使用 \pkg{algorithms} 宏包或者较新的 \pkg{algorithm2e} 实现。
算法~\ref{algo:algorithm} 是一个使用 \pkg{algorithm2e} 的例子。关于排版算法环境
的具体方法,请阅读相关宏包的官方文档。

\begin{algorithm}[htb]
  \caption{算法示例}
  \label{algo:algorithm}
  \small
  \SetAlgoLined
  \KwData{this text}
  \KwResult{how to write algorithm with \LaTeXe }

  initialization\;
  \While{not at end of this document}{
    read current\;
    \eIf{understand}{
      go to next section\;
      current section becomes this one\;
    }{
      go back to the beginning of current section\;
    }
  }
\end{algorithm}

\section{代码环境}

我们可以在论文中插入算法,但是不建议插入大段的代码。如果确实需要插入代码,建议使
用 \pkg{listings} 宏包。

\begin{codeblock}[language=C]
#include <stdio.h>
#include <unistd.h>
#include <sys/types.h>
#include <sys/wait.h>

int main() {
  pid_t pid;

  switch ((pid = fork())) {
  case -1:
    printf("fork failed\n");
    break;
  case 0:
    /* child calls exec */
    execl("/bin/ls", "ls", "-l", (char*)0);
    printf("execl failed\n");
    break;
  default:
    /* parent uses wait to suspend execution until child finishes */
    wait((int*)0);
    printf("is completed\n");
    break;
  }

  return 0;
}
\end{codeblock}
