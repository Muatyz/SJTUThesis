% !TEX root = ../main.tex

\chapter{散射尾波}

\section{超声波在颗粒介质中的扩散近似}

Weaver R. L. 推导出了超声波在多晶体中的扩散行为方程\cite{diffusivity}:

\begin{equation}
  \frac{\partial I}{\partial t} - D\nabla^{2}I + \frac{I}{\tau_{\alpha}} = \delta(z)\delta(t)
\end{equation}


\section{超声波在颗粒介质中的非线性}

\subsection{非线性波方程与谐波推导}

已知含黏度 $\eta$ 的小振幅声波方程为

\begin{equation}
  \rho_{0}\frac{\partial^{2}\xi}{\partial t^{2}} - \rho_{0}c_{0}\frac{\partial^{2}\xi}{\partial x^{2}} - \eta\frac{\partial^{3}\xi}{\partial x^{2}\partial t} = 0.
\end{equation}

其中 $\xi$ 是声位移,$\rho_{0}$ 与 $c_{0}$ 分别是介质在平衡态下的密度与声速。引入声速的非线性 $c = c_{0} + \beta\frac{\partial\xi}{\partial t}$,同时注意到信号速度

\begin{equation}
  c_{0} = \left(\frac{\partial \xi}{\partial t}\right)/\left(\frac{\partial\xi}{\partial x}\right)
\end{equation}

于是得到非线性 Burgers 波方程:

%% 这里需要找到 Hamiton Blackstock 1999 的参考文献
%% 以确认推导的过程基本正确

\begin{equation}
  \rho_{0}\frac{\partial^{2}\xi}{\partial t^{2}} - \eta\frac{\partial^{3}\xi}{\partial x^{2}\partial t} - \rho_{0}c_{0}^{2}\frac{\partial^{2}\xi}{\partial x^{2}} - 2\rho_{0}c_{0}^{2}\beta\left(\frac{\partial \xi}{\partial x}\right)\left(\frac{\partial^{2} \xi}{\partial x^{2}}\right) = 0
\end{equation}



\subsection{相似性参数}

引入相似性参数 $\Psi_{i,j}(\tau)$, 以描述信号 $S_{i}$ 与 $S_{j}$ 的相似程度\cite{PhysRevLett.90.174302}:

\begin{equation}
  \Psi_{i,j}(\tau) = \frac{\int_{-\infty}^{+\infty}S_{i}(t)S_{j}(t+\tau)\mathrm{d}t}{\sqrt{\int_{-\infty}^{+\infty}S_{i}^{2}(t)\mathrm{d}t\int_{-\infty}^{+\infty}S_{j}^{2}(t)\mathrm{d}t}}.
\end{equation}

由于信号已被归一化,所以得到的将是 $[-1,1]$ 之间的数值。相似性参数越接近 $1$,则两个信号越相似,即颗粒介质对 $S_{i}$ 与 $S_{j}$ 的散射效果越接近,从而证明颗粒介质的力链结构变化越小。事实上,这种相似性参数对于颗粒介质内部的应力结构变化极为敏感,这也是低振幅超声波被认定为颗粒介质探针的证据之一。

\subsection{声源振幅对颗粒介质可逆性的影响}