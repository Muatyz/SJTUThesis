% !TEX root = ../main.tex

\chapter{绪论}

\section{颗粒物质}

“当一个物体被分作几个单独运动的小部分时,它是液态的;而当它的所有部分都接触在一起时,它是固态的。” 这就是 17 世纪笛卡尔对于颗粒物质机械性质的朴素认知。颗粒物质在自然界中广泛存在,拥有类似于寻常物质的固(颗粒静堆积)、液(颗粒流)、气(颗粒分布均匀、快速运动)三态\cite{RevModPhys.68.1259},其共通特征是离散物体的聚集系统,如沙漠、土壤、泥石流、沙尘暴;而在人类文明中,其无疑也占有重要地位,如谷物堆积、矿石输运、药物粉末等生产行为。即使是被称作低廉材料的煤炭、水泥、沙砾等物质,其生产与加工对地球上生产能源消耗的比例也来到了约 \num{10}\%\cite{duran2000sands}。因此对于颗粒物质的研究,不仅有益于进一步认知自然界,从而防治地震、滑坡、雾霾、沙漠化等自然灾害,还有助于提升人类工业文明的生产效率、提高生产质量。

颗粒物质的物理学是复杂的学科。颗粒物质通常被定义为尺度大于微米量级的宏观离散体,通过接触、碰撞等形式聚集组成的多体系统。其特性如下:

\begin{enumerate}
  \item 宏观性。杜绝了量子层面的隧穿效应与原子层面的热涨落效应;在颗粒介质中无需考虑隧穿效应等量子效应。
  \item 离散性。使得传统的连续介质力学对其的描述存在偏差\cite{RevModPhys.71.435};
  \item 多体性。三体问题已经是能产生混沌现象的复杂力学体系,而颗粒介质的多体特征注定了对其进行宏观力学建模求解时需要面对的极高自由度困难;
  \item 耗散性。颗粒间复杂的摩擦与碰撞使得系统中的成员能迅速将动能转化为分子层面的热运动。因此颗粒系统中存在强耗散性,在失去外部能量驱动的情况下可以长期处于某一亚稳态。常规的统计力学假设,即系统遍历相空间,对于颗粒介质失效。颗粒介质是非平衡态的系统;
  \item 敏感性。颗粒介质中存在高度复杂的力链与接触网络,而外部激励容易使得颗粒进行重排(Rearrangement),使网络具有高敏感性,因此对于颗粒介质的任何研究方法都应当慎重考虑技术细节。
\end{enumerate}

颗粒物质的物理学也是年轻的学科。2005 年 Science 提出的 125 个最重要前沿科学问题\cite{doi:10.1126/science.309.5731.78b}中,“能否发展关于湍流动力学和颗粒材料运动学的综合理论?”赫然在列;而 2021 年提出的 “新 125 个科学问题”\cite{sanders2021125}中,“集体运动的基本原理是什么?”仍然彰显着颗粒物质物理学的神秘与活力。

\section{颗粒物质的研究方法现状}


\subsection{实验方法}

针对颗粒物理中丰富的物理性质,目前已开发出多角度的实验研究方法。

\begin{enumerate}
  \item 光-弹性实验\cite{photoelasticimetry}。对于透明介质,其内部应变时表现出双折射现象,因此可以直观体现颗粒介质在受单轴剪切、环形剪切等条件下其中内的应力分布。但是由于偏振光在异质性介质中的传播、三维介质中的精确测量困难等问题\cite{Non-Destructive_3D_Photoelasticity},这种方法通常局限于二维平面内;
  \item X 射线扫描建模三维结构\cite{PhysRevE.68.020301}。X 射线扫描图像经由去边界、分水岭算法分割等处理后重现颗粒物质的三维结构。但是该方法通常难以应用于实地实验。CT 设备成本较高,且即使单张图像所需的存储空间也已较大(对于图像分辨率 $1021\times 1021 \times 886$,约占 $1.7\unit{\giga\byte}$),因此对于实验时的多帧图像进行数据处理还需要配备价格不菲的计算资源。X 射线存在衍射极限而空间分辨率存在上限,因此对于所研究颗粒的半径、材料都存在一定限制;
  \item 核磁共振\cite{CLARKE2023}。另一类对颗粒介质进行成像研究的手段,只是“光源”换为了磁场。相应的,颗粒介质中需要对应的标记原子(原子核存在非零净磁矩,如最常用的质子成像就需要 H$^{1}$),搭建对应的磁场线圈等也存在一定的技术限制。
\end{enumerate}

以及本文所着眼的超声探测技术。超声波在小振幅时具有探测作用,有限振幅则对颗粒介质具有诱导重排、流化、软化等泵浦作用。

\begin{enumerate}
  \item 主动探测。超声波在颗粒介质中的传播是对其内部应力与接触结构的直接体现\cite{PhysRevB.48.15646,Jia1999UltrasoundPI,Transitional}。具有非侵入性、高灵敏度等优点,真实世界中的颗粒材料通常光学不透明或者剧烈光散射,此时超声波成为颗粒物质研究中独一无二的探针。对经由颗粒介质传播后的声信号进行频谱、波形、声速、衰减、相似性参数等分析,从而研究颗粒介质内部的结构特征;
  \item 被动探测。颗粒介质受剪切时出现应力-应变曲线,伴随着内部颗粒的重排、碰撞与摩擦等过程,这些过程会产生颗粒介质的机械振动,即声信号。这种事件被称为声发射(Acoustic Emission,AE)。分析 AE 信号的频谱、能量、波形、振动模密度等特征\cite{PhysRevLett.120.218003,10.1029/2023JB026612,doi:10.1073/pnas.2305134120,}可以对颗粒介质受应力时的滞滑(Stick-slip)事件等特征进行研究。
\end{enumerate}


\subsection{理论方法}

虽然颗粒物质是非平衡态系统,但是对其进行统计力学上的描述仍在争取中。在实验中已经发现,通过振台振动、剪切盒循环剪切等外部激励方式操纵的颗粒堆积,其存在可重复的微观态;若采取等概率假设,即相同体积颗粒堆积的微观结构概率相同,那么就可以将体积替代历史上统计力学中的哈密顿量,从而建立颗粒物质的统计力学框架。Edwards 以这种思想提出了“等效温度”的定义:

\begin{equation}
  \frac{1}{\chi} = \frac{\partial S(V)}{\partial V} = \frac{\partial \lambda\ln{\Omega(V)}}{\partial V},
\end{equation}

其中 $\chi$ 为等效温度(正式变量名为 Compactivity),$S$ 为熵,$\Omega(V)$ 是给定堆积体积 $V$ 下的微观态数目,$\lambda$ 为类比玻尔兹曼常数 $k_{B}$ 的常数。已有实验证明,这种定义的等效温度与通过涨落耗散效应定义的温度具有一致性。

%%% 插图。主要是说明振台振动和剪切盒的实验装置

对于声学而言,依靠的主要是非线性声学\cite{10.1029/93JB02974}、等效介质理论(Effective Media Theory,EMT),扩散行为近似。
%%% 这里需要一个在颗粒介质中的声信号示例图

图~\ref{fig:reference_point} 展示了在颗粒固体中的传播示意图。可以观察到,超声波通常由相干首波(Coherent Wave,简略为 E)与散射尾波(Codalike Scatter Wave,简略为 S)。 %%% 有关相干信号自平均的讨论

在连续介质力学中我们已经知道,三维介质中的波传播分为两种模式,即纵波/压缩波(Longitudinal Wave/Compressional Wave),以及横波/剪切波(Transverse Wave/Shear Wave)。这两种模式波引入到颗粒介质中时,即对应体弹性模量 $K$ 与剪切模量 $G$。EMT 理论综合了颗粒介质特有的体积分数 $\phi$、平均接触数/配位数 $Z$ 等参数与连续介质中的弹性模量的定义,从而预测颗粒介质中不同模式的声速等物理量。

声学对于颗粒介质的研究范式大致可以根据等效波长 $\lambda_{\text{eff}} = V_{\text{eff}}/f_{c}$ 与振幅 $A$ 来进行总结分类。记颗粒直径为 $d$。

\begin{itemize}
  \item $\lambda_{E}\gg d$。此时关心的是在颗粒介质中的相干弹性波(coherent elastic waves)。由于此时颗粒介质的力学响应可以类比于寻常固体,所以 EMT 采用仿射变换近似(affine approximation)、平均场近似等方法将其处理为等效的连续介质。此时关心的是通常是横纵波速 $V_{P}$、$V_{S}$ 与等效介质弹性模量 $K$、$G$ 之间存在的关系。
  \item $\lambda_{E}\sim d$。此时声波在颗粒介质中的传播表现出强烈的多重散射特征。颗粒接触点的耗散性使得声波在颗粒介质中的衰减吸收大大加强了,因此提出了扩散行为近似\cite{PhysRevLett.93.154303}、非线性波方程\cite{Transitional,hamilton_nonlinear_1998}等模型来对颗粒介质进行研究。品质因子 $Q$、平均自由程 $l^{*}$ 等参数用于研究颗粒接触点的耗散性。
  \item $A\rightarrow 0$。对于小振幅声源,颗粒介质的非线性仅体现为其等效黏度 $\eta$ 等介质固有特征,不会对颗粒介质的接触与应力结构造成显著影响。在此范围内我们将超声波视为非侵入性探针;
  \item $A\rightarrow \delta$。对于有限振幅声源,颗粒介质的非线性将被进一步激发,其等效黏度会根据超声波振幅而产生变化,即声源已经对颗粒介质造成泵浦效果。比如借助超声波诱导倾斜的颗粒介质发生滑坡(landslide,avalanche)\cite{PhysRevE.102.042901},地震波诱导的二次地震(co-seismic)\cite{Johnson_2005}等现象。
\end{itemize}

