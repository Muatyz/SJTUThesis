% !TEX root = ../main.tex

\chapter{绪论}

\section{颗粒物质}

“当一个物体被分作几个单独运动的小部分时,它是液态的;而当它的所有部分都接触在一起时,它是固态的。” 这就是 17 世纪笛卡尔对于颗粒物质的朴素认知。颗粒物质在自然界中广泛存在,其共通特征是离散的聚集系统,如沙漠、土壤、泥石流;而在人类文明中,其无疑也占有重要地位,如谷物堆积、矿石输运、药物粉末等生产行为。即使是被称作低廉材料的煤炭、水泥、沙砾等物质,其生产与加工对地球上生产能源消耗的比例也来到了约 \num{10}\%\cite{duran2000sands}。因此对于颗粒物质的研究,不仅有益于进一步认知自然界,防治地震、滑坡、雾霾、沙漠化等自然灾害,还有助于提升人类工业文明的生产效率、提高生产质量。

颗粒物质的物理学是复杂的学科。颗粒物质被定义为尺度大于微米量级的宏观离散体,通过接触、碰撞等形式聚集组成的多体系统。宏观性杜绝了量子层面的隧穿效应与原子层面的热涨落效应,却又因为离散性使得传统的连续介质力学对其的描述存在偏差\cite{RevModPhys.71.435};多体特征注定了对颗粒物质建模求解时高自由度的困难,且颗粒间复杂的摩擦与碰撞令颗粒介质具有强耗散性,从而进一步加剧了这种困难;依赖于接触网络与力链网络的颗粒介质具有对外部激励的高敏感性,因此展开对颗粒介质的研究时更需要考虑周全技术细节。

颗粒物质的物理学也是年轻的学科。2005 年 Science 提出的 125 个最重要前沿科学问题\cite{doi:10.1126/science.309.5731.78b}中,“能否发展关于湍流动力学和颗粒材料运动学的综合理论?”赫然在列;而 2021 年提出的 “新 125 个科学问题”\cite{sanders2021125}中,“集体运动的基本原理是什么?”仍然彰显着颗粒物质物理学的神秘与活力。

\section{颗粒物质的声学研究方法}

虽然颗粒介质是相当年轻的学科,但是目前已经发展出种类繁多的研究方法, 如光弹性实验,X 射线建模三维结构,以及本文所着眼的超声波探测技术。


