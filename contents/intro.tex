% !TEX root = ../main.tex

\chapter{绪论}

\section{颗粒物质}

“当一个物体被分作几个单独运动的小部分时,它是液态的;而当它的所有部分都接触在一起时,它是固态的。” 这就是 17 世纪笛卡尔对于颗粒物质的朴素认知。颗粒物质在自然界中广泛存在,其共通特征是离散的聚集系统,如沙漠、土壤、泥石流;而在人类文明中,其无疑也占有重要地位,如谷物堆积、矿石输运、药物粉末等生产行为。即使是被称作低廉材料的煤炭、水泥、沙砾等物质,其生产与加工对地球上生产能源消耗的比例也来到了约 \num{10}\%\cite{duran2000sands}。因此对于颗粒物质的研究,不仅有益于进一步认知自然界,防治地震、滑坡、雾霾、沙漠化等自然灾害,还有助于提升人类工业文明的生产效率、提高生产质量。

颗粒物质的物理学是复杂的学科。颗粒物质被定义为尺度大于微米量级的宏观离散体,通过接触、碰撞等形式聚集组成的多体系统。宏观性杜绝了量子层面的隧穿效应与原子层面的热涨落效应,却又因为离散性使得传统的连续介质力学对其的描述存在偏差\cite{RevModPhys.71.435};多体特征注定了对颗粒物质建模求解时高自由度的困难,且颗粒间复杂的摩擦与碰撞令颗粒介质具有强耗散性,从而进一步加剧了这种困难;依赖于接触网络与力链网络的颗粒介质具有对外部激励的高敏感性,因此展开对颗粒介质的研究时更需要考虑周全技术细节。

颗粒物质的物理学也是年轻的学科。2005 年 Science 提出的 125 个最重要前沿科学问题\cite{doi:10.1126/science.309.5731.78b}中,“能否发展关于湍流动力学和颗粒材料运动学的综合理论?”赫然在列;而 2021 年提出的 “新 125 个科学问题”\cite{sanders2021125}中,“集体运动的基本原理是什么?”仍然彰显着颗粒物质物理学的神秘与活力。

\section{颗粒物质的声学研究方法}

即使颗粒物理是相当年轻的学科,目前也已经发展出种类繁多的研究方法,如光弹性实验\cite{photoelasticimetry},X 射线扫描建模三维结构\cite{PhysRevE.68.020301},以及本文所着眼的超声探测技术。光弹性实验通过光学方法可以直观体现颗粒介质内的应力分布,但是由于偏振光在异质性介质中的传播、三维介质中的精确测量困难等问题\cite{Non-Destructive_3D_Photoelasticity},这种方法通常局限于二维平面;X 射线扫描建模可以研究颗粒物质的三维结构,但是通常难以应用于实地实验,对于颗粒材料有一定要求,图像分割的分辨率存在上限等。而超声波在颗粒介质中的传播是对其内部应力结构的直接体现,具有非侵入性、高灵敏度等优点,因此也被应用于颗粒物质的研究中。

声学对于颗粒介质的研究范式大致可以根据其有效波长 $\lambda_{\text{eff}} = V_{\text{eff}}/f_{c}$ 与振幅 $A$ 来进行分类。记颗粒直径为 $d$。

\begin{itemize}
  \item $\lambda_{E}\gg d$。此时关心的是在颗粒介质中的相干弹性波(coherent elastic waves)。由于此时颗粒介质的力学响应可以类比于寻常固体,所以采用仿射变换近似(affine approximation)、平均场近似等方法将其处理为等效的连续介质。此时关心的是通常是横纵波速 $V_{P}$、$V_{S}$与等效介质弹性模量 $K$,$G$ 之间存在的关系。
  \item $\lambda_{E}\sim d$。此时声波在颗粒介质中的传播表现出强烈的多重散射特征。颗粒接触点的耗散性使得声波在颗粒介质中的衰减吸收加强了,因此提出了扩散行为近似\cite{PhysRevLett.93.154303}、非线性波方程\cite{Transitional,hamilton_nonlinear_1998}等模型来对颗粒介质进行研究。品质因子 $Q$、平均自由程 $l^{*}$ 等参数用于研究颗粒接触点的耗散性。
  \item $A\rightarrow 0$。对于小振幅声源,颗粒介质的非线性仅体现为其等效黏度 $\eta$ 等介质固有特征,不会对颗粒介质的接触与应力结构造成显著影响。在此范围内我们将超声波视为非侵入性探针
  \item $A\rightarrow \delta$。对于有限振幅声源,颗粒介质的非线性将被进一步激发,其等效黏度由于振幅而发生变化,即声源已经对颗粒介质造成激励与重排效果。比如超声波诱导的倾斜的颗粒介质滑坡\cite{PhysRevE.102.042901},地震波诱导的二次地震\cite{johnson_nonlinear_2005}等现象。
\end{itemize}

