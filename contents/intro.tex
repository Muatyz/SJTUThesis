% !TEX root = ../main.tex

\chapter{绪论}

\section{颗粒物质}

“当一个物体被分作几个单独运动的小部分时,它是液态的;而当它的所有部分都接触在一起时,它是固态的。” 这就是 17 世纪笛卡尔对于颗粒物质的朴素认知。颗粒物质在自然界中广泛存在,其共通特征是离散的聚集系统,如沙漠、土壤、泥石流;而在人类文明中,其无疑也占有重要地位,如谷物堆积、矿石输运、药物粉末等生产行为。即使是被称作低廉材料的煤炭、水泥、沙砾等物质,其生产与加工对地球上生产能源消耗的比例也来到了约 \num{10}\%\cite{duran2000sands}。因此对于颗粒物质的研究,不仅有益于进一步认知自然界,防治地震、滑坡、雾霾、沙漠化等自然灾害,还有助于提升人类工业文明的生产效率、提高生产质量。

颗粒物质的物理学是复杂的学科。颗粒物质被定义为尺度大于微米量级的宏观离散体,通过接触、碰撞等形式聚集组成的多体系统。宏观性杜绝了量子层面的隧穿效应与原子层面的热涨落效应,却又因为离散性使得传统的连续介质力学对其的描述存在偏差\cite{RevModPhys.71.435};多体特征注定了对颗粒物质建模求解时高自由度的困难,且颗粒间复杂的摩擦与碰撞令颗粒介质具有强耗散性,从而进一步加剧了这种困难;依赖于接触网络与力链网络的颗粒介质具有对外部激励的高敏感性,因此展开对颗粒介质的研究时更需要考虑周全技术细节。

颗粒物质的物理学也是年轻的学科。2005 年 Science 提出的 125 个最重要前沿科学问题\cite{doi:10.1126/science.309.5731.78b}中,“能否发展关于湍流动力学和颗粒材料运动学的综合理论?”赫然在列;而 2021 年提出的 “新 125 个科学问题”\cite{sanders2021125}中,“集体运动的基本原理是什么?”仍然彰显着颗粒物质物理学的神秘与活力。

\section{颗粒物质的声学研究方法}

虽然颗粒介质是相当年轻的学科,但是目前已经发展出种类繁多的研究方法, 如光弹性实验,X 射线建模三维结构,以及本文所着眼的超声波探测技术。


\section{脚注}

Lorem ipsum dolor sit amet, consectetur adipiscing elit, sed do eiusmod tempor
incididunt ut labore et dolore magna aliqua. \footnote{Ut enim ad minim veniam,
quis nostrud exercitation ullamco laboris nisi ut aliquip ex ea commodo
consequat. Duis aute irure dolor in reprehenderit in voluptate velit esse cillum
dolore eu fugiat nulla pariatur.}

\section{字体}



{\songti 十九世纪末,甲午战败,民族危难。中国近代著名实业家、教育家盛宣怀和一批
  有识之士秉持“自强首在储才,储才必先兴学”的信念,于 1896 年在上海创办了交通大
  学的前身——南洋公学。}

{\heiti 改革开放以来,学校以“敢为天下先”的精神,大胆推进改革:率先组成教授代
  表团访问美国,率先实行校内管理体制改革,率先接受海外友人巨资捐赠等,有力地推动
  了学校的教学科研改革。1984 年,邓小平同志亲切接见了学校领导和师生代表,对学校
  的各项改革给予了充分肯定。}

{\ifcsname fangsong\endcsname\fangsong\else[无 \cs{fangsong} 字体。]\fi 交通大学
  始终把人才培养作为办学的根本任务。一百多年来,学校为国家和社会培养了 20余万各
  }

{\ifcsname kaishu\endcsname\kaishu\else[无 \cs{kaishu} 字体。]\fi 截至 2011 年 12
 }
