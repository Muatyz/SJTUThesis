% !TEX root = ../main.tex

\chapter{绪论}

\section{颗粒物质}

“当一个物体被分作几个单独运动的小部分时,它是液态的;而当它的所有部分都接触在一起时,它是固态的。” 这就是 17 世纪笛卡尔对于颗粒物质机械性质的朴素认知。颗粒物质在自然界中广泛存在,其共通特征是离散物体的聚集系统,如沙漠、土壤、泥石流;而在人类文明中,其无疑也占有重要地位,如谷物堆积、矿石输运、药物粉末等生产行为。即使是被称作低廉材料的煤炭、水泥、沙砾等物质,其生产与加工对地球上生产能源消耗的比例也来到了约 \num{10}\%\cite{duran2000sands}。因此对于颗粒物质的研究,不仅有益于进一步认知自然界,从而防治地震、滑坡、雾霾、沙漠化等自然灾害,还有助于提升人类工业文明的生产效率、提高生产质量。

颗粒物质的物理学是复杂的学科。颗粒物质通常被定义为尺度大于微米量级的宏观离散体,通过接触、碰撞等形式聚集组成的多体系统。其特性如下:

\begin{enumerate}
  \item 宏观性。杜绝了量子层面的隧穿效应与原子层面的热涨落效应;
  \item 离散性。使得传统的连续介质力学对其的描述存在偏差\cite{RevModPhys.71.435};
  \item 多体性。三体问题已经是能产生混沌现象的复杂力学体系,而颗粒介质的多体特征注定了对其进行宏观力学建模求解时需要面对的极高自由度困难;
  \item 耗散性。颗粒间复杂的摩擦与碰撞使得系统中的成员能迅速将动能转化为分子层面的热运动,因此颗粒介质具有强耗散性,在失去外部驱动的情况下可以长期处于某一亚稳态。因此常规的统计力学假设,即系统遍历相空间,对于颗粒介质失效;
  \item 敏感性。颗粒介质中存在高度复杂的力链与接触网络,而外部激励容易使得颗粒进行重排,使网络具有高敏感性,因此对于颗粒介质的任何研究方法都应当慎重考虑技术细节。
\end{enumerate}

颗粒物质的物理学也是年轻的学科。2005 年 Science 提出的 125 个最重要前沿科学问题\cite{doi:10.1126/science.309.5731.78b}中,“能否发展关于湍流动力学和颗粒材料运动学的综合理论?”赫然在列;而 2021 年提出的 “新 125 个科学问题”\cite{sanders2021125}中,“集体运动的基本原理是什么?”仍然彰显着颗粒物质物理学的神秘与活力。由于颗粒介质,因此在 

\section{颗粒物质的研究方法}

由于颗粒物质本身具有高度复杂的结构与力学特性,人类对于其展开的研究如同盲人摸象,所开发的丰富研究方法展现了人类智慧与勇攀高峰的决心。

\begin{enumerate}
  \item 光弹性实验\cite{photoelasticimetry}。光弹性实验通过光学方法可以直观体现颗粒介质内的应力分布,但是由于偏振光在异质性介质中的传播、三维介质中的精确测量困难等问题\cite{Non-Destructive_3D_Photoelasticity},这种方法通常局限于二维平面;
  \item X 射线扫描建模三维结构\cite{PhysRevE.68.020301}。X 射线扫描建模可以研究颗粒物质的三维结构,但是通常难以应用于实地实验,对于颗粒材料有一定要求(不能是金属材料),图像分割的分辨率存在上限等;
  \item 超声探测技术。超声波在颗粒介质中的传播是对其内部应力与接触结构的直接体现,具有非侵入性、高灵敏度等优点,因此也被应用于颗粒物质的研究中。
\end{enumerate}


%%% 这里需要一个在颗粒介质中的声信号示例图

图~\ref{fig:reference_point} 展示了在颗粒固体中的传播示意图。

声学对于颗粒介质的研究范式大致可以根据等效波长 $\lambda_{\text{eff}} = V_{\text{eff}}/f_{c}$ 与振幅 $A$ 来进行总结分类。记颗粒直径为 $d$。

\begin{itemize}
  \item $\lambda_{E}\gg d$。此时关心的是在颗粒介质中的相干弹性波(coherent elastic waves)。由于此时颗粒介质的力学响应可以类比于寻常固体,所以采用仿射变换近似(affine approximation)、平均场近似等方法将其处理为等效的连续介质。此时关心的是通常是横纵波速 $V_{P}$、$V_{S}$与等效介质弹性模量 $K$、$G$ 之间存在的关系。
  \item $\lambda_{E}\sim d$。此时声波在颗粒介质中的传播表现出强烈的多重散射特征。颗粒接触点的耗散性使得声波在颗粒介质中的衰减吸收大大加强了,因此提出了扩散行为近似\cite{PhysRevLett.93.154303}、非线性波方程\cite{Transitional,hamilton_nonlinear_1998}等模型来对颗粒介质进行研究。品质因子 $Q$、平均自由程 $l^{*}$ 等参数用于研究颗粒接触点的耗散性。
  \item $A\rightarrow 0$。对于小振幅声源,颗粒介质的非线性仅体现为其等效黏度 $\eta$ 等介质固有特征,不会对颗粒介质的接触与应力结构造成显著影响。在此范围内我们将超声波视为非侵入性探针
  \item $A\rightarrow \delta$。对于有限振幅声源,颗粒介质的非线性将被进一步激发,其等效黏度由于振幅而发生变化,即声源已经对颗粒介质造成激励与重排效果。比如超声波诱导的倾斜的颗粒介质滑坡\cite{PhysRevE.102.042901},地震波诱导的二次地震\cite{johnson_nonlinear_2005}等现象。
\end{itemize}

