% !TEX root = ../main.tex

\chapter{全文总结}

颗粒介质中的超声信号一般可分为相干首波和散射尾波,小振幅时具有探测作用,有限振幅时则具有泵浦作用。对于相干波部分,我们通过飞行时间法和频散能量图法测定了颗粒介质中的声速,并且检查了声速和颗粒介质应力之间的指数关系,分析了其与等效介质理论之间差异的可能原因;观察到了首波的展宽现象,通过归一化宽度对其进行表述,其在不同厚度的颗粒介质中的值与通过一维随机层理论导出的预测符合较好。对散射波部分,我们通过截断级数拟合功率时域图,计算得到平均自由程和品质因子,从而检验了扩散行为近似导出的辐射传递方程的符合较好;通过增大激励信号的幅值,检验了颗粒介质中的非线性波激发现象,并且通过相似性参数和声速观察到了声波对于颗粒介质产生的不可逆泵浦作用。在对颗粒介质进行剪切时,观察到了 RLP 和 RCP 的不同应力-应变曲线,并且通过滞滑事件计数进行了滞滑失效统计预测的现象学分析。而利用相似性参数检测颗粒介质中的剪切带和监测受剪切过程的声发射事件有待后续研究。