% !TEX root = ../main.tex

\begin{abstract}[zh]
  颗粒介质是常见的强耗散非平衡态体系。目前对于颗粒介质的研究手段中,CT 成像等方式虽然已经成熟,但是不擅长于处理动态问题,而声学的高灵敏、非侵入特性能够弥补这一点。
  从熟悉学习颗粒介质中的声学研究方法出发,本文研究了超声波在颗粒介质中的传播特性。
  振幅较小时,超声波可作为颗粒介质内部应力结构的探针;振幅较大时,超声波则作为一种外部激励的能源。
  典型的颗粒介质中的声信号可视为两部分:相干弹性波和多重散射尾波。
  对于相干弹性波,本文讨论了如何使用飞行时间法和频散能量图法测定声速,并且将其和传统的时差法进行了比较;观察到了首峰随传播距离增大出现的展宽现象,使用归一化宽度对其进行描述并且和一维随机层理论导出的结果符合较好;检测了声速和介质所受单轴应力之间的指数关系,并与等效介质理论的预测进行对比和误差分析。
  对于散射尾波,本文探讨了由辐射传递方程导出的扩散行为近似,并且发现其与测定的强度曲线符合较好,拟合计算得到了平均自由程和非弹性吸收时间;研究了振幅和颗粒介质非线性的关系,发现振幅超过一定值后使得其结构因泵浦而变化,并引入相似性参数对其进行描述。
  最后对颗粒介质的剪切响应进行了研究,发现了无法向应力的随机稀疏堆积和施加法向应力的随机密集堆积应力-应变曲线的不同,并且对后者进行了微滞滑的现象学分析,得出了预测介质失效的图像。
  这些技术积累将为后续使用声学研究颗粒介质中的剪切带、监测受剪切过程的声发射事件以及将声学方法耦合至 CT 成像研究等提供基础。
\end{abstract}

\begin{abstract}[en]
  Granular media are strongly dissipative nonequilibrium systems. Currently, although methods like CT imaging for granular media research are already mature, they are not adept at handling dynamic problems, while the high sensitivity and non-invasive nature of acoustic method can compensate for this. 
  To be familiar with acoustic research methods in granular matter, this paper investigates the propagation characteristics of ultrasonic waves in granular media. 
  When the amplitude is small, ultrasonic waves can serve as probes for the internal stress structure of granular media; when the amplitude is large, ultrasonic waves act as a source of external excitation. 
  Typical acoustic signals in granular media can be treated as two component: coherent elastic waves and multiple-scattered tail waves. 
  For coherent elastic waves, this paper discusses how to measure the speed of sound using time-of-flight method and dispersion energy map method, and compares it with traditional time-difference method; the phenomenon of broadening of the first peak with increasing propagation distance is observed, and its description using normalized width is consistent with the results derived from one-dimensional random layer theory; the exponential relationship between sound speed and uniaxial stress on the medium is detected and compared with the predictions of effective medium theory, along with error analysis. 
  For scattered tail waves, this paper discusses the diffusion behavior approximated by the radiative transfer equation and finds good agreement with the measured intensity curve, fitting to calculate the mean free path and non-elastic absorption time; the relationship between amplitude and nonlinearity of granular media is studied, finding that exceeding a certain value of amplitude causes configuration changes due to pumping and introduces the similarity parameter for description. 
  Finally, the shear response of granular media is studied, revealing the differences between stress-strain curves of random loose packing under no stress and random close packing under normal stress, and a phenomenological analysis of micro-slip is conducted on the latter, yielding a predictive image of medium failure. This accumulation of acoustic techniques will provide a foundation for subsequent studies using acoustics to probe shear bands in granular media, monitor acoustic emission events during shear processes, and couple acoustic methods to CT imaging research, among others.
\end{abstract}